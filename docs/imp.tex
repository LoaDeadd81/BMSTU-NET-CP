\chapter{Конструкторский раздел}

В данном разделе будут представлены технические аспекты реализации программы.

\section{Средства реализаци}

Для реализации ПО был выбран язык C~\cite{clang} так как он был указан в задании и обладает всем необходимым функционалом для реализации требуемого программного обеспечения. В качестве среды разработки была выбрана Clion~\cite{clion}.

\section{Реализация сервера}

В листинге \ref{lst:run} представленная реализация функции запуска сервера.

\begin{center}
\captionsetup{justification=raggedright,singlelinecheck=off}
\begin{lstlisting}[label=lst:run,caption=Запуск сервера]
int run_http_server_t(http_server_t *server) {
	log_info("Server starting");
	log_info("Host: %s", server->host);
	log_info("Port: %d", server->port);
	log_info("Work dir: %s", server->wd);
	
	server->listen_sock = listen_net(server->host, server->port);
	if (server->listen_sock < 0) return -1;
	
	if (run_tpool_t(server->pool) != 0) return -1;
	
	long maxi = 0, nready;
	server->clients[0].fd = server->listen_sock;
	server->clients[0].events = POLLIN;
	
	while (1) {
		nready = poll(server->clients, maxi + 1, -1);
		if (nready < 0) {
			log_fatal(ERR_FSTR, "poll error", strerror(errno));
			return -1;
		}
		
		if (server->clients[0].revents & POLLIN) {
			int client_sock = accept_net(server->listen_sock);
			if (client_sock < 0) continue;
			
			long i = 0;
			for (i = 1; i < server->cl_num; ++i) {
				if (server->clients[i].fd < 0) {
					server->clients[i].fd = client_sock;
					break;
				}
			}
			if (i == server->cl_num) {
				log_error("too many connections");
				continue;
			}
			server->clients[i].events = POLLIN;
			
			if (i > maxi) maxi = i;
			if (--nready <= 0) continue;
		}
		for (int i = 1; i <= maxi; ++i) {
			if (server->clients[i].fd < 0) continue;
			
			if (server->clients[i].revents & (POLLIN | POLLERR)) {
				task_t *task = new_task_t(handle_connection, server->clients[i].fd, server->wd);
				if (task == NULL) continue;
				
				add_task(server->pool, task);
				server->clients[i].fd = -1;
				
				if (--nready <= 0) break;
			}
		}
	}
}
\end{lstlisting}
\end{center}
\FloatBarrier

В листинге \ref{lst:routine} представленная реализация функции воркера thread pool, в которой он дожидается поступления задания в очередь, забирает его и запускается обработку запроса. Само задание представляет собой структуру хранящую указатель на функцию обработки запросов и указатели на параметры для неё. Она представленная в листинге \ref{lst:task}.

\begin{center}
\captionsetup{justification=raggedright,singlelinecheck=off}
\begin{lstlisting}[label=lst:routine,caption=Получение задачи из очереди]
void *routine(void *args) {
	routine_args_t *r_args = (routine_args_t *) args;
	tpool_t *pool = r_args->pool;
	int num = r_args->num;
	free(args);
	
	task_t task;
	char name[15] = "";
	sprintf(name, "thread-%d", num);
	thread_name = name;
	
	while (1) {
		pthread_mutex_lock(pool->q_mutex);
		
		s_wait(pool->sem, pool->q_mutex);
		
		if (pool->stop == 1) {
			pthread_mutex_unlock(pool->q_mutex);
			log_debug("stopped");
			break;
		}
		
		int rc = pop(pool->queue, &task);
		if (rc != -1) {
			log_debug("work taken (q len = %d)", pool->queue->len);
		}
		pthread_mutex_unlock(pool->q_mutex);
		if (rc < 0) continue;
		
		task.handler(task.conn, task.wd);
		log_info("routine for task finished");
	}
	
	pthread_exit(NULL);
}
\end{lstlisting}
\end{center}

\begin{center}
\captionsetup{justification=raggedright,singlelinecheck=off}
\begin{lstlisting}[label=lst:task,caption=Получение задачи из очереди]
typedef struct routine_args_t {
	tpool_t *pool;
	int num;
} routine_args_t;
\end{lstlisting}
\end{center}
\FloatBarrier

В листинге \ref{lst:handler} представленная реализация функции обработки HTTP запроса.

\begin{center}
\captionsetup{justification=raggedright,singlelinecheck=off}
\begin{lstlisting}[label=lst:handler,caption=Обработка запроса]
void handle_connection(int clientfd, char *wd) {
	request_t req;
	char *buff = calloc(REQ_SIZE, sizeof(char));
	if (buff == NULL) {
		log_error(ERR_FSTR, "failed alloc req buf", strerror(errno));
		return;
	}
	
	log_debug("handle_connection started");
	if (read_req(buff, clientfd) < 0) {
		send_err(clientfd, INT_SERVER_ERR_STR);
		close(clientfd);
		free(buff);
		return;
	}
	log_debug("read_req");
	
	if (parse_req(&req, buff) < 0) {
		send_err(clientfd, BAD_REQUEST_STR);
		close(clientfd);
		free(buff);
		return;
	}
	if (req.method == BAD) {
		log_error("unsupported http method");
		send_err(clientfd, M_NOT_ALLOWED_STR);
		close(clientfd);
		free(buff);
		return;
	}
	
	process_req(clientfd, &req, wd);
	
	close(clientfd);
	log_debug("handle_connection finished");
	
	free(buff);
}
\end{lstlisting}
\end{center}
