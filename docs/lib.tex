\addcontentsline{toc}{chapter}{СПИСОК ИСПОЛЬЗОВАННЫХ ИСТОЧНИКОВ}

\renewcommand\bibname{СПИСОК ИСПОЛЬЗОВАННЫХ ИСТОЧНИКОВ}
\begin{thebibliography}{}

\setlength\bibindent{1.25cm}
\makeatletter
\let\old@biblabel\@biblabel
\def\@biblabel#1{\kern\bibindent\old@biblabel{#1}}
\makeatother

\bibitem{seti} Джеймс Куроуз, Кит Росс Компьютерные сети: Нисходящий подход. --- 6-e изд. --- М.: Издательсво <<Э>>, 2016. --- 912 C. 

\bibitem{conn} Connection management in HTTP/1.x [Электронный ресурс]. --- URL: \url{https://developer.mozilla.org/ru/docs/Web/HTTP/Connection_management_in_HTTP_1.x} (дата обращения: 12.12.2023).

\bibitem{tpool} Энтони Уильямс Параллельное программирование на C++ в действии. Практика разработки многопоточных программ. --- М.: ДМК Пресс, 2012. --- 672 C. 

\bibitem{clang} The GNU C Reference Manual [Электронный ресурс]. --- URL: \url{https://www.gnu.org/software/gnu-c-manual/gnu-c-manual.html} (дата обращения: 12.12.2023).

\bibitem{clion} CLion Кросс-платформенная IDE для C и C++ [Электронный ресурс]. --- URL: \url{https://www.jetbrains.com/ru-ru/clion/} (дата обращения: 12.12.2023).

\bibitem{linux} Manjaro Linux [Электронный ресурс]. --- URL: \url{https://manjaro.org/} (дата обращения: 12.12.2023).

\bibitem{cpu} Процессор Intel® Core™ i5-7300HQ [Электронный ресурс]. --- URL: \url{https://ark.intel.com/content/www/ru/ru/ark/products/97456/intel-core-i5-7300hq-processor-6m-cache-up-to-3-50-ghz.html} (дата обращения: 12.12.2023).

\end{thebibliography}


